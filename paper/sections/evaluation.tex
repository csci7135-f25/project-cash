In this section, we examine two case studies of the cumulative abstract semantics: one realized in a native language with built-in support for algebraic effects and lexical handlers, Effekt ~\cite{Schuster2022}, and another realized via type classes in a more conventional functional language, Lean. Both implementations adhere to an identical conceptual scheme and yield extensionally equivalent analyses, thereby enabling a comparison in terms of runtime performance and relative implementation complexity.

Our target language is a Python-like language, whose syntax is presented in Figure \ref{fig:python-syntax}.
% 
\begin{figure}
    \centering
    \begin{mathparsmall}
    \begin{array}{lrcl}
        \text{Identifiers} & x \\
        \text{Numbers} & n & \in & \mathbb{R}\\
        \text{Booleans} & b & := & \text{true}\mid \text{false}\\
        \text{Concrete Values} & v & := & n \mid b \mid ()\\
        \text{Operators} & op & := & + \mid - \mid \div \mid *\mid == \mid > \mid < \mid \text{and} \mid \text{or}\\
        \text{Expressions} & e & := & \text{cst}(v) \mid \text{var}(x) \mid \text{binop}(e, e, op) \mid \text{neg}(e)\\
        \text{Statements} & s & := & \text{skip} \mid \text{assign}(x, e) \mid \text{if}(e, s, s) \mid \text{seq}(s, s) \mid \text{while}(e, s) \\
        \text{Program node} & p & := & s \mid e
    \end{array}
    \end{mathparsmall}
    \caption{A python-like AST}
    \label{fig:python-syntax}
\end{figure}
% % 
% To evaluate the complexity associated with extending the language with new features, we introduce a single additional language feature in both versions after having made an interpreter:
% % 
% \begin{mathparsmall}
%     \begin{array}{lrcl}
%         \text{Expressions} & e & := & \cdots \mid \text{namedExpr}(x, e) \\
%     \end{array}
% \end{mathparsmall}
% % 
% The extended language would now include the two forms of assignment present in Python: (i) the traditional statement-level assignment and (ii) the “walrus” expression-level assignment (namedExpr). The latter behaves like a C assign statement where it mutates the store and returns the value assigned. With this, we can compare the number of lines of code required to extend the language accordingly.


\subsection{Typeclass Based}\label{sec:eval-lean}
The cumulative semantics can be defined with type classes  relying on the language's type resolution capabilities to select appropriate instances at compile time. To clarify nomenclature, an interface in this version will be a type class, while a witness will be an instance of that class. We select Lean4 to make this interpreter for several reasons: It is a clean, modern functional programming language that is growing in popularity, and compiles to performant C binaries. Additionally Lean provides the ability to reason about programs. The capability to carry proofs with our semantics opens the door to reusable soundness proofs, among others. Another practical advantage is with Lean one can get develop-time error checking if there is not an appropriate type class instance, eliminating any runtime errors where these handler may have been missing only to be discovered during testing.

\subsubsection{Type Class Architecture}
The unsubstantiated evaluator establishes the foundation for all subsequent instantiations. Its signature contains the set of elimination handlers that must be provided to produce an executable interpreter:
\begin{center}
    \begin{lstlisting}[language=lean]
partial def eval {σ δ : Type}
  [CstE σ δ] [VarE σ δ] [BinopE σ δ] [NegE σ δ]
  [SkipE σ δ] [AssignE σ δ] [IfE σ δ] [SeqE σ δ] [WhileE σ δ]
  : Prog → σ → (δ × σ)
\end{lstlisting}
\end{center}
% 
This evaluator remains parametric over both the state type $\sigma$ and the abstract domain $\delta$. The interpreter implements a state-passing denotational semantics, reflected in its return type of $\sigma \to \delta \times \sigma$. This function itself performs a straightforward structural recursion over the syntax, delegating to the listed typeclasses. The expression case shows this clearly:
% 
\begin{lstlisting}[language=lean]
| (.Exp e) => match e with
  | .Cst v => CstE.cst eval v
  | .Binop e1 e2 op => BinopE.binop eval e1 e2 op
  | .Neg e => NegE.neg eval e
\end{lstlisting}
% 
Each syntatic construct invokes its corresponding elimination handler, passing the evaluator as the first argument to enable recursion. By passing itself, it maintains the same scope that it was called in, rather than defaulting to the global scope. This pattern extends to all language constructs.

\subsubsection{Handler Signature Patterns}
The type signature for handler pairs follow systematic patterns as established in Section \ref{sec:tech-pattern}. We examine representative examples from each of the three categories to illustrate how these patterns manifest via typeclasses.

Computational nodes, like binary operations, operate purely on domain values without modifying state. In Lean the signature looks like the following:
% 
\begin{lstlisting}[language=lean]
class BinopE (σ δ : Type) where
  binop : (Prog → σ → (δ × σ)) → Expr → Expr → Op → σ → (δ × σ)
class BinopI (δ : Type) where
  binop : δ → δ → Op → δ
\end{lstlisting}
% 
The elimination handler(BinopE) accepts a continuation for evaluation, the raw syntactic parameters, and the current state. It promises to return an updated state-value pair. The introduction interface operates only on domain values. Each expression in its parameter list becomes a domain value $\delta$, while the operator remains unchanged.

Syntax nodes that interact with state are demonstrated by the \emph{assign} statement:
% 
\begin{lstlisting}[language=lean]
class AssignE (σ δ : Type) where
  assign : (Prog → σ → (δ × σ)) → Ident → Expr → σ → (δ × σ)
class AssignI (σ δ : Type) where
  assign : Ident → δ → σ → (δ × σ)
\end{lstlisting}
% 
Unlike computational nodes, the introduction interface for assignment must accept a state, and returns a state-value tuple. Similar to computation nodes, state nodes transform expressions into domain values.

Control flow nodes transform statements into first class state transformations. This is why we refer to this implementation as somewhat denotational: because the evaluation is broken into state transformations that are then threaded together. Their introduction handlers accept the initial state, thread it through evaluation, returning only a state. The \emph{sequence} handler illustrates this approach:
% 
\begin{lstlisting}[language=lean]
class SeqE (σ δ : Type) where
  seq : (Prog → σ → (δ × σ)) → Stmt → Stmt → σ → (δ × σ)
class SeqI (σ δ : Type) where
  seq: σ → (σ → σ) → (σ → σ) → σ
\end{lstlisting}
% 
The introduction interface receives the initial state along with two state transformers, representing the semantic functions for each statement. This design allows the introduction witness to determine how state flows through sequential compositions: whether strictly left-to-right, with potential short-circuiting, or through some other control flow strategy—without being coupled to the specific syntax of the statements involved.

\subsubsection{Concrete Evaluation}
Substantiating the interpreter entails providing an instance for each required typeclass. We again examine binary operations, through the implementation of their witnesses: 
\begin{lstlisting}[language=lean]
instance {σ δ : Type} [BinopI δ] : BinopE σ δ where
  binop eval e1 e2 op ρ :=
    let (v1, ρ') := eval (.Exp e1) ρ
    let (v2, ρ'') := eval (.Exp e2) ρ'
    (BinopI.binop v1 v2 op, ρ'')
    
instance : BinopI ConcreteValue where
  binop v1 v2 op := match (v1, v2, op) with
    | (.Num n1, .Num n2, Op.Plus) => .Num (n1 + n2)
    | (.Num n1, .Num n2, Op.Minus) => .Num (n1 - n2)
    ...
\end{lstlisting}
% 
The elimination instance remains generic over some state $\sigma$ and domain $\delta$, while the introduction instance is tied to the $ConcreteValue$ domain. This genericity allows the same $BinopE$ handler to be used across many domains simply by providing a new $BinopI$ instance. In a similar fashion, we implement the Assign handlers:
\begin{lstlisting}[language=lean]
instance {σ δ : Type} [AssignI σ δ] : AssignE σ δ where
  assign eval x e ρ :=
    let (v, ρ') := eval (.Exp e) ρ
    AssignI.assign x v ρ'
    
instance {σ δ : Type} [Inhabited δ] [Put σ δ] : AssignI σ δ where
  assign x v ρ := (default, Put.put x v ρ)
\end{lstlisting}
% 
The elimination handler, as typical, decides the threading of state by evaluating the right hand side, then passing that state to the introduction handler. The introduction handler then modifies this state with the Put type class, which encapsulates the actual state update operation. This type class can be thought of as a lowering handler, preventing the need for reimplementation of assignment logic. This additional abstraction permits flexible state representations like simple maps, more sophisticated scoped states, or even relational states. The $Inhabited$ type class, which allows the elimination handler to return a default value (unit) without knowing the domain, can also be thought of as a lowering handler. Something of note here is these specific instances are generic over any domain, and so could be reused for abstract interpretation.

Control flow handlers demonstrate how state transformations are assembled from syntax and packaged. This is clear in the conditional handler pair:
% 
\begin{lstlisting}[language=lean]
instance {σ δ : Type} [Inhabited δ] [IfI σ δ] [Assume σ δ] : IfE σ δ where
  if_ eval e t f ρ :=
    let (v, ρ') := eval (.Exp e) ρ
    let tk : σ → σ := λ σ => Assume.assume v (eval (.Stm t) σ).snd
    let fk : σ → σ := λ σ => Assume.assumef v (eval (.Stm f) σ).snd
    (default, IfI.if_  ρ' tk fk)
    
instance {σ δ : Type}  [Join σ] : IfI σ δ where
  if_ ρ tk fk :=
    (tk ρ) ⊔ (fk ρ)
\end{lstlisting}
% 
The elimination handler constructs two state transforms: one that assumes the guard condition and evaluates the true branch, and another that assumes the negation and evaluates the false branch. It also evaluates the guard condition. With the new state and the two state transforms, it calls the introduction handler which determines how to join the two. This implementation shows a join of the two branches, giving us flow sensitive analysis. If we wanted to do a flow insensitive analysis it would be as simple as supplying a different elimination handler that has no assume calls in it.

The while loop presents the most complex control flow scenario, requiring fixpoint iteration:
% 
\begin{lstlisting}[language=lean]
partial def lfp {α : Type} (f : α → α) (x : α) [Bottom α] [LatOrder α] : α :=
  let rec aux (current : α) :=
    let next := f current
    if next ⊑ current then current else aux next
  aux x

instance {σ δ : Type} [Assume σ δ] [WhileI σ δ] [Inhabited δ] : WhileE σ δ where
  while_ eval e body ρ:=
    let k : σ → σ := fun σ =>
      let (v, σ') := eval (.Exp e) σ
      Assume.assume v (eval (.Stm body) σ').snd
    let invariant := WhileI.while_ ρ k
    let final := eval (.Exp e) invariant
    (default, Assume.assumef final.1 final.2)

instance{σ δ: Type} [Bottom σ] [LatOrder σ]: WhileI σ δ where
  while_ ρ cont := lfp cont ρ
\end{lstlisting}
% 
The elimination handler constructs a state transform for one iteration of the loop: evaluate the guard, then execute the body and return the result with an assume. The introduction handler applies fixpoint iteration to this transform, starting from the inital state and continuing until convergence. The least fixpoint implementation requires that the state forms a lattice with bottom and ordering operations, which are both constraints captured in the type classes $Bottom$ and $LatOrder$. This design successfully separates the syntactic structure of loops from the semantic strategy for computing their effect, allowing the same elimination handler to support different fixpoint strategies or widening operators through alternative introduction instances.

% \subsubsection{Modifying the language}
% To evaluate the extensibility of this framework, we extended the language with python's walrus operator, or named expressions. This required modification in four locations: the grammar of the language, defining elimination and introduction type classes, the generic evaluator, and implementing the two handler instances. It is hypothesized that the first three of these could be done automatically. The complete extension totals approximately 18 lines of code:
% \begin{lstlisting}[language=lean]
% -- Grammar.lean (2 lines)
% | NamedExpr: Ident → Expr → Expr

% -- Classes.lean (6 lines)
% class NamedExprE (σ δ : Type) where
%   namedExpr : (Prog → σ → (δ × σ)) → Ident → Expr→ σ → (δ × σ)
% class NamedExprI (σ δ : Type) where
%   namedExpr : Ident → δ → σ → (δ × σ)

% -- Eval.lean (2 line)
% [NamedExpreE σ δ]
% ...
% | .NamedExpr x e => NamedExprE.namedExpr eval x e

% -- Concrete.lean (7 lines)
% instance {σ δ : Type} [NamedExprI σ δ] : NamedExprE σ δ where
%   namedExpr eval x e σ :=
%     let (v, σ') := eval (.Exp e) σ
%     NamedExprI.namedExpr x v σ'
% instance {σ δ : Type} [Put σ δ] : NamedExprI σ δ where
%   namedExpr x v σ := (v, Put.put x v σ)
% \end{lstlisting}
% % 
% Critically, no existing code was lost, and no semantics were duplicated. This contrasts to the monolithic evaluator where all semantic code was duplicated in the new version of the evaluator.

% The type class approach provides several advantages as an implementation of cumulative semantics. First, the compiler enforces completeness at compile time, so lacking instantiations are not discovered at runtime. Second, the explicit type class constraints serve as machine-checked documentation of an interpreter's requirements, making dependencies between components transparent and verifiable. These properties position the type class approach as particularly suitable for building and maintaining large collections of program analyses where correctness and maintainability are paramount concerns. Third, type classes in lean also can contain proofs, which opens the door to carry soundness proofs with each semantic chunk if the developer desired.
% 
% \subsubsection{Abstract Interpretation}
% Currently, we have not finished the implementation for an abstract interpreter but we are confident that it would be possible.

\subsection{Effect Based}\label{sec:eval-effekt}
\subsubsection{Handlers and Interfaces}
As explored in our introductory work on Cumulative Semantics, the theory is cleanly expressed with algebraic effects and handlers ~\cite{Plotkin_2013}. The key insight being that effect handlers offer an implicit resume, eliminating the need for passing a continuation with every recursive call. Any language with an effect system supporting generic effect interfaces and multiple resumptions can express the ethos of cumulative semantics.

Previously we implemented our analyzer with Koka, but found that due to the need for several handlers with multiple resumptions performance dropped significantly. The Koka compiler is not optimized for such a task. Instead, we decided to implement with Effekt ~\cite{Schuster2022} due to its efficient compilation of effect handlers to CPS style code.

Due to the high level of abstraction provided by algebraic effects, implementing an analyzer that adheres to cumulative semantics with a direct style is extremely natural.

The subset of Python's Syntax is represented as an ADT named $Prog$, of two smaller ADT's. One for expressions, and one for statements. Elimination handlers are defined as an effect interface, parametrized by a generic domain $D$. They take in state, a piece of syntax, and or an $Ident$, represented by a $String$. They return a tuple of $Prog$ and state, signaling that their handlers reduce or eliminate syntax while updating state. Due to the branching nature of syntax ($if$, $seq$, etc..) elimination handlers utilize multiple resumptions to pass each syntactic branch back into the evaluation function. As a result the elimination effects are grouped into a single interface, ensuring that when a new, processed syntax is resumed, the handler of this syntax is known immediately.

\begin{lstlisting}[language=scala]
interface Elimination[D] {
  def varE(st: State[D], x: String): (Prog, State[D])
  ...
  def seqE(st: State[D], e1: Stmt, e2: Stmt): (Prog, State[D])
}
\end{lstlisting}

Additionally introduction interfaces are expressed with effects. Algebraic effects are powerful enough to emulate type classes. They are parameterized by the same generic domain type $D$, and will only resume once, allowing them to be defined individually. Introduction interfaces are tasked with introducing new values and states of a syntactic domain, as demonstrated by their type signatures.

\begin{lstlisting}[language=scala]
effect plusI[D](st: State[D], d1: D, d2: D): (D, State[D]) 
effect asgnI[D](st: State[D], x: String, v: D): (D, State[D])
\end{lstlisting}

\subsubsection{Unsubstantiated Evaluation} Once the elimination and introduction effects have been typed it is possible to create an unsubstantiated evaluation function. The type signature denotes the presence of the $Elimination$ effect interface, as well as any introduction handlers for constant domain values. These can either be expressed as elimination handlers or directly as introduction handles as shown below. The $eval$ function takes in a program and initial state, returning a generic domain value and state parameterized by that generic domain. The $do$ notation calls an effect, the resumption of which will return a processed piece of syntax back into the recursive $eval$ function call.

\begin{lstlisting}[language=scala]
def eval[D](prog: (Prog, State[D])): (D, State[D]) /{
    intI[D], boolI[D], unitI[D], varI[D],
    Elimination[D] } = 
  val (e, st) = prog
  e match {
    case E(e) => e match
      case Plus(e1, e2) => eval(do plusE(st, e1, e2))
      ...
\end{lstlisting}

\subsubsection{Concrete Evaluation} There are two steps to achieving an executable evaluation. First we must substantiate the elimination Effects for concrete values. Then intro effects can be substantiated to produce values in the domain of the current evaluation. The syntax below expresses a function that wraps an effectful one, providing handlers to a specified effect. This wrapper can introduce new effects, even if they are called by another effect's handlers. Each elimination handler calls its corresponding introduction interface, which we account for in the type of the function.

\begin{lstlisting}[language=scala]
def match_to_fold
  {prog: => (Val, State[Val]) / {Elimination[Val]}}:
  (Val, State[Val]) / {
    ...
    ifI[Val], whileI[Val], seqI[Val] } =
    { try {prog()}
      with Elimination[Val] {
        def plusE(st, e1, e2) =
          val (v1, st1) = resume((E(e1), st))
          val (v2, st2) = resume((E(e2), st1))
          do plusI(st2, v1, v2)
        ...
\end{lstlisting}

While the return type of the handler must match the type passed to its $resume$, it can be thought of as a returning a continuation to the recursive call of $eval$. Once evaluation has reached a leaf of the AST,the elimination handler delegates returning domain values to its introduction interface, passing in any updated values and states.

Evaluation becomes executable once all elimination and introduction effects are handled. Now with the elimination effects handled, a wrapper to give semantics to introduction interfaces fully substantiates $eval$ is instantiated.

\begin{lstlisting}[language=scala]
def run(prog: Prog, init: State[Val]): (Val, State[Val]) = {
  try {
    with match_to_fold()
    eval((prog, init))}
  ...
  with ifI[Val] {(stT, stF, g, t, f) =>
    g match {
      case B(true) => resume((t, stT))
      case B(false) => resume((f, stF))
      case _ => resume((U(), stF))
\end{lstlisting}

Introduction handlers choose the values and states a syntactic node should return. This behavior pins the domain of evaluation and directionality of control flow. As seen in the witness (effect handler) for $if$, the witness receives two states for each branch of the if, the value of evaluating the guard condition, the true branch, and the false branch. It then dictates which branch's state should be returned based on the guard value. Herein lies the value for relational analysis. The introduction handler can decide to combine the states of diverging branches instead of choosing only one.

The Effekt implementation of cumulative semantics is very explicit and due to the direct style facilitated by algebraic effects, the theory is not lost within the implementation. Both elimination handlers and introduction witnesses are expressed with the same construct creating a congruence that allows the developer to focus more on the theory of cumulative semantics over the implementation itself.

\subsection{Comparison of the Two Implementations}\label{sec:eval-comparison}
% 
To evaluate performance, we executed a simple while-loop program with varying iterations using the native Python interpreter, as well as the Lean4 and Effekt interpreters. Execution times were measured with the hyperfine benchmarking tool. The results are reported in Table \ref{tab:benchmark_results}. Although both cumulative semantic implementations introduce a non-negligible overhead, their performance remains within one order of magnitude of the Python interpreter, indicating reasonably good scaling behavior. The nearly constant run time of the Lean4 implementation for small input sizes suggests a substantial fixed overhead, likely associated with type-class resolution, but the subsequent growth in running time is moderate. In contrast, the Effekt-based implementation exhibited markedly superior performance, not yet exploding.
% 
 \begin{table}[th!]
    \centering
    \begin{tabular}{llcccc}
        \toprule
        \textbf{Implementation} & \textbf{Iterations ($N$)} & \textbf{Time (Mean $\pm$ $\sigma$)} & \textbf{User} & \textbf{System} & \textbf{Range} \\
        \midrule
        \multirow{3}{*}{Effekt LLVM} & 100,000 & $64.8 \text{ ms} \pm 1.2 \text{ ms}$   & $64.3 \text{ ms}$ & $0.4 \text{ ms}$ & $61.5 \text{ ms} \dots 66.8 \text{ ms}$ \\
         & 10,000 & $7.5 \text{ ms} \pm 0.1 \text{ ms}$    & $7.2 \text{ ms}$  & $0.2 \text{ ms}$ & $7.2 \text{ ms} \dots 8.3 \text{ ms}$ \\
         & 1,000 & $1.7 \text{ ms} \pm 0.1 \text{ ms}$    & $1.4 \text{ ms}$  & $0.2 \text{ ms}$ & $1.6 \text{ ms} \dots 2.4 \text{ ms}$ \\
        \midrule
        \multirow{3}{*}{Lean4} & 100,000 & $126.7 \text{ ms} \pm 326.8 \text{ ms}$ & $20.1 \text{ ms}$ & $5.0 \text{ ms}$ & $22.2 \text{ ms} \dots 1056.7 \text{ ms}$ \\
         & 10,000 & $22.1 \text{ ms} \pm 0.3 \text{ ms}$   & $18.0 \text{ ms}$ & $1.9 \text{ ms}$ & $21.5 \text{ ms} \dots 23.2 \text{ ms}$ \\
         & 1,000 & $22.1 \text{ ms} \pm 0.3 \text{ ms}$   & $18.0 \text{ ms}$ & $1.9 \text{ ms}$ & $21.4 \text{ ms} \dots 22.8 \text{ ms}$ \\
        \midrule
        \multirow{3}{*}{Python} & 100,000 & $17.4 \text{ ms} \pm 1.5 \text{ ms}$   & $13.1 \text{ ms}$ & $3.1 \text{ ms}$ & $16.6 \text{ ms} \dots 31.3 \text{ ms}$ \\
         & 10,000 & $14.1 \text{ ms} \pm 0.3 \text{ ms}$   & $9.9 \text{ ms}$  & $3.1 \text{ ms}$ & $13.6 \text{ ms} \dots 16.3 \text{ ms}$ \\
         & 1,000 & $13.7 \text{ ms} \pm 0.3 \text{ ms}$   & $9.6 \text{ ms}$  & $3.1 \text{ ms}$ & $13.3 \text{ ms} \dots 14.9 \text{ ms}$ \\
        \bottomrule
    \end{tabular}
    \caption{Benchmark Comparison of Loop Iterations}
     \label{tab:benchmark_results}
\end{table}
% 