In this paper, we have developed an intuitive account of the purpose of elimination, introduction, and lowering instances/handlers. We subsequently formalized and refined these intuitions, presenting a systematic methodology for constructing the corresponding semantics. This methodology was then instantiated in two distinct programming languages with slightly different design structures. The resulting variations illustrate the expressiveness and flexibility of the proposed framework, demonstrating that it can be used in multiple ways to accommodate different interpretation styles.  Our performance evaluation indicates that the framework incurs some overhead; however, in relative terms this overhead remains moderate and does not render the approach impractical.

Cumulative abstract semantics, as presented here, are subject to several limitations, the most significant being the lack of comprehensive validation. In particular, there is currently no complete implementation for abstract interpretation; existing implementations are restricted to concrete interpreters.

To address these limitations, we envision several directions for future work. First, we plan to construct abstract interpreters in both object languages, supporting interval analysis and backward symbolic weakest-precondition analysis. In addition, we aim to establish correctness results in the presence of monadic state, which would in turn enable a principled treatment of error handling via monads and related constructs. Finally, it would be valuable to investigate, within Lean, methodologies for structuring and transporting soundness proofs so that they can be reused across different sound components.