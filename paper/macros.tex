% envs
\newenvironment{mathparsmall}
  {\begin{displaysmall}\begin{mathpar}}
  {\end{mathpar}\end{displaysmall}}
\newenvironment{displaysmall}
  {\begin{center}\begin{minipage}{0.96\linewidth}\centering\small}
  {\end{minipage}\end{center}}
\newenvironment{grammar}
  {\begin{mathpar}\begin{array}{rr@{}c@{}lrl}}
  {\end{array}\end{mathpar}}

% colors and fonts
\newcommand{\syntaxcolor}[1]{{\color{Mahogany}#1}}
\newcommand{\introcolor}[1]{{\color{Fuchsia}#1}}
\newcommand{\elimcolor}[1]{{\color{Emerald}#1}}
\newcommand{\loweringcolor}[1]{{\color{Periwinkle}#1}}

\newcommand{\syntax}[1]{{\syntaxcolor{\texttt{#1}}}}
\newcommand{\intro}[1]{{\introcolor{\grave{#1}}}}
\newcommand{\elim}[1]{{\elimcolor{\acute{#1}}}}
\newcommand{\meta}[1]{{\color{OliveGreen}\texttt{#1}}}
\newcommand{\lowering}[1]{{\loweringcolor{\grave{#1}}}}

\newcommand{\unsubst}[1]{\Breve{#1}}

% evaluation 
% operational
\newcommand{\judge}[1]{\boxed{#1}}
\newcommand{\evalopsym}{\Downarrow}
\newcommand{\evalopnostate}[3]{#1 \evalopsym_{#2} #3}
\newcommand{\evalop}[4]{#1 \vdash #2 \evalopsym_{#3} #4}
% denotational
\newcommand{\eval}[1]{\left\llbracket #1 \right\rrbracket}
\newcommand{\evalemp}{\eval{\cdot}}
\newcommand{\unsubsteval}[1]{\unsubst{\eval{#1}}}
\newcommand{\evalnostate}[3]{\eval{#1}_{#2} = #3}
\newcommand{\unsubstevalnostate}[3]{\unsubsteval{#1}_#2 = #3}
\newcommand{\unsubstevalstate}[4]{\unsubsteval{#1}_#2 = \lambda #3.#4}
\newcommand{\evalabs}[1]{\eval{#1}^\sharp}
\newcommand{\collecting}[1]{\left\{ \mkern-4.5mu \left\llbracket #1 \right\rrbracket \mkern-4.5mu \right\}}

% language macros
\newcommand{\expr}{e}
\newcommand{\stmt}{s}
\newcommand{\prog}{p}
\newcommand{\const}{v}
\newcommand{\intC}{n}
\newcommand{\boolC}{b}
% symbols
\newcommand{\cstsym}{cst}
\newcommand{\addsym}{+}
\newcommand{\ifnzsym}{if}
\newcommand{\varsym}{var}
\newcommand{\asgnsym}{:=}
% syntax
\newcommand{\cst}[1]{\syntax{\cstsym(}#1\syntax{)}}
\newcommand{\add}[2]{#1\:\syntax{\addsym}\: #2}
\newcommand{\ifnz}[3]{\syntax{\ifnzsym\;(}#1\syntax{)\:} #2 \syntax{\:else\:} #3}
\newcommand{\var}[1]{\syntax{\varsym(}#1\syntax{)}}
\newcommand{\asgn}[2]{#1\syntax{\;\asgnsym\;}#2}
\newcommand{\skp}{\syntax{skip}}
\newcommand{\seq}[2]{#1 \syntax{;} #2}
% introduction
\newcommand{\cstI}[1]{\intro{\cstsym(}#1\introcolor{)}}
\newcommand{\addI}[2]{#1\:\intro{\addsym}\: #2}
\newcommand{\ifnzI}[3]{\intro{\ifnzsym\;(}#1\introcolor{)\:} #2 \introcolor{\:else\:} #3}
\newcommand{\varI}[1]{\intro{\varsym(}#1\introcolor{)}}
\newcommand{\asgnI}[2]{#1\intro{\;\asgnsym\;}#2}
% eliminiation
\newcommand{\cstE}[1]{\elim{\cstsym(}#1\elimcolor{)}}
\newcommand{\addE}[2]{#1\:\elim{\addsym}\: #2}
\newcommand{\ifnzE}[3]{\elim{\ifnzsym\;(}#1\elimcolor{)\:} #2 \elimcolor{\:else\:} #3}
\newcommand{\varE}[1]{\elim{\varsym(}#1\elimcolor{)}}
\newcommand{\asgnE}[2]{#1\elim{\;\asgnsym\;}#2}

% abstract domain operators
\newcommand{\joinsym}{\sqcup}
\newcommand{\joinLsym}{\lowering{\joinsym}}
\newcommand{\assumesym}{assume}
\newcommand{\assumensym}{{\assumesym}n}
\newcommand{\assumeLsym}{\lowering{\assumesym}}
\newcommand{\assumenLsym}{\lowering{\assumensym}}
\newcommand{\joinfmt}[3]{#1\;#2\;#3}
\newcommand{\joinL}[2]{\joinfmt{#1}{\joinLsym}{#2}}
\newcommand{\assumefmt}[3]{#1 \loweringcolor{(}#2\loweringcolor{)} #3}
\newcommand{\assumeL}[2]{\assumefmt{\assumeLsym}{#1}{#2}}
\newcommand{\assumenL}[2]{\assumefmt{\assumenLsym}{#1}{#2}}

% extraneous
\newcommand{\tuple}[2]{\left\langle #1, #2 \right\rangle}
\newcommand{\absval}{\mathscr{d}}
\newcommand{\absstore}{\sigma}

\newcommand{\effect}[1]{\backslash \left\langle #1 \right\rangle}

\newcommand{\metalet}[3]{\meta{let} #1 \:\meta{=}\: #2 \meta{ in } #3}

\newcommand{\rec}{\texttt{rec}}